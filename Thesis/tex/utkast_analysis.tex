\subsection{Analysis} %Etter torsdagsprat med Neri angåande glmer, glmer og CoxPH, kva eg har og korleis det skal presenterast

To find out I used a GLMM with the glmer function from R package lme4.
I used n.obs as the response variable, and time since deployment interacting with flash as the predictors.
Location ID and week of the year were used as Random effects to account for differences between the camera sites and seasonal changes.
I made one model for each species.

However, this model only takes into account whether a flash was present or not. It can't tell if the flash actually went off in any way.

Therefore I set up a new column called flashed, that told if the flash went off in syncrony with the IR camera.



I used this information in a Cox PH model with the coxme package to account for the random effects.
If a species was flashed, it went into the "flashed"-treatment group, and time to next detection was recorded. 
If the species didn't reappear it was "censored" from the model.

Both these models tell me something about the fallacy of $H_0$, whether I can reject it, or fail to reject it.

If any of these to models are significant it would give me reason to reject the null hypothesis, and then go from H1 to consider the H2 hypothesis.
However, if none of them are significant, there still could be confounding effects due to the differences in urbanisation between sites. %språkvask naudsynleg

So, to test my last hypothesis --- is there an urbanisation effect on the species' reaction to white LED --- I used a new Cox PH-model, this time with the coxph-function from the R package Survival.
To look for these spatial patterns I need to preserve the variation between cameras to allow my spatial fixed effects to account for the variation.
I used distance to nearest house (data from TK) as a proxy for urbanisation and Artificial Light at Night (ALAN) and I used microhabitat / trail type (recorded at each site by NINA) to account for the variation between these structures, which is quite large.

The response-variable was - as before - a survival object (Surv(Time, status)), and the predictor variables were "flashed" interacting with distance to nearest house, plus habitat/trail type.
