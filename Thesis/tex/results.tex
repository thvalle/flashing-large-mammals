\chapter{Results}
%\subsection{General results}
%
%After excluding faulty cameras we had TK cameras in total, 19 in the Control-group and 33 in the experiment group.
%A total of TK photos were taken, of whom TK contained photos of the species I've focused on in my study. Check table for more details.
%
%The species I've focused on was mainly night-active as displayed in the density plots \ref{fig:overlap}, (with the exception of squirrel). In other words, all of them experienced a white LED flash during night hours.
%(Caption: Figure is split up in periods with and without white LED flash. Night activity in the dashed curve highlights the times the species would have experienced the flash. "Carpet"-below the curves signifies datapoints of each "group".
%



\subsection{GLMM}

For roe deer, the model explaining variation in detection rate has a total explanatory power that is substantial (conditional R2 = 0.45), but the part related to the fixed effects alone (marginal R2) is just 0.01.

In other words, most of the variation in detection rate is due to seasonal changes and variation between the different camera sites captured in the random terms.
Usage of white LED over time had no significant effect on detection rate of roe deer.

All three treatment groups (flash 0 / 1 / Control) showed a negative, non-significant trend over time.
As the control-group stayed unchanged through the whole study period, and was visited less than the other cameras, I expected there to be no trend over time (i.e. time.deploy:flashControl $\approx 0$).

Any fluctuations in detection rates due to weekly (and ultimately seasonal) changes should be controlled for by the random effect-argument for week of the year.
%A negative slope for the control group is strange, as it should represent a baseline detection probability.

Nevertheless, the equivalence test in figure X %\ref{fig:para_raa3}
indicate that the negative slopes are only due to chance, as all parameters related to time since deployment are well within the Region of Practical Equivalence (ROPE).



When a parameter is within the ROPE in an equivalence test, it signifies that the difference from the mean, and the variance of the parameter, is low enough that we can accept H0.
According to this test, white LED is different enough that we cannot conclude on it’s immediate effect (intercept value), but it’s trend over time (interaction with time since deployment) is practically equivalent to H0.
In other words, I haven't found any effect from the usage of white LED on the detection rate of roe deer, and fail to reject H0.




\input{tex/fig/result_figs.tex}


% latex table generated in R 4.0.3 by xtable 1.8-4 package
% Wed Mar 03 10:42:32 2021
\begin{table}[ht]
\centering
\begin{tabular}{rllllll}
  \hline
 & Parameter & Coefficient & SE & 95\% CI & z & p \\ 
  \hline
1 & Roe deer &  &  &  &  &        \\ 
  2 & (Intercept) & -3.49 & 0.29 & (-4.06, -2.91) & -11.84 & $<$ .001 \\ 
  3 & time.deploy & -0.06 & 0.06 & (-0.17,  0.05) & -1.04 & 0.297  \\ 
  4 & flash [1] & 0.05 & 0.07 & (-0.09,  0.18) & 0.64 & 0.522  \\ 
  5 & flash [Control] & -0.16 & 0.50 & (-1.14,  0.82) & -0.32 & 0.748  \\ 
  6 & time.deploy * flash [1] & -0.04 & 0.07 & (-0.17,  0.10) & -0.56 & 0.572  \\ 
  7 & time.deploy * flash [Control] & 0.03 & 0.08 & (-0.12,  0.18) & 0.41 & 0.681  \\ 
  8 & Red fox &  &  &  &  &        \\ 
  9 & (Intercept) & -3.41 & 0.17 & (-3.74, -3.09) & -20.50 & $<$ .001 \\ 
  10 & time.deploy & -1.87e-03 & 0.07 & (-0.13,  0.13) & -0.03 & 0.978  \\ 
  11 & flash1 & 0.12 & 0.09 & (-0.05,  0.29) & 1.36 & 0.174  \\ 
  12 & flashControl & -0.04 & 0.28 & (-0.59,  0.50) & -0.16 & 0.872  \\ 
  13 & time.deploy:flash1 & -0.02 & 0.09 & (-0.19,  0.15) & -0.23 & 0.815  \\ 
  14 & time.deploy:flashControl & -1.21e-03 & 0.09 & (-0.19,  0.18) & -0.01 & 0.990  \\ 
  15 & Badger &  &  &  &  &        \\ 
  16 & (Intercept) & -4.26 & 0.29 & (-4.83, -3.69) & -14.65 & $<$ .001 \\ 
  17 & time.deploy & 0.14 & 0.08 & (-0.01,  0.29) & 1.87 & 0.062  \\ 
  18 & flash1 & 0.06 & 0.09 & (-0.12,  0.24) & 0.62 & 0.534  \\ 
  19 & flashControl & -0.35 & 0.39 & (-1.10,  0.41) & -0.89 & 0.371  \\ 
  20 & time.deploy:flash1 & -2.65e-03 & 0.09 & (-0.18,  0.18) & -0.03 & 0.977  \\ 
  21 & time.deploy:flashControl & -0.11 & 0.11 & (-0.34,  0.11) & -0.99 & 0.324  \\ 
  22 & Moose &  &  &  &  &        \\ 
  23 & (Intercept) & -4.62 & 6.93e-04 & (-4.62, -4.61) & -6661.58 & $<$ .001 \\ 
  24 & time.deploy & 0.11 & 6.93e-04 & ( 0.11,  0.11) & 155.60 & $<$ .001 \\ 
  25 & flash1 & 0.15 & 6.93e-04 & ( 0.15,  0.15) & 219.58 & $<$ .001 \\ 
  26 & flashControl & -0.19 & 6.93e-04 & (-0.19, -0.19) & -275.38 & $<$ .001 \\ 
  27 & time.deploy:flash1 & -0.14 & 6.93e-04 & (-0.14, -0.14) & -203.48 & $<$ .001 \\ 
  28 & time.deploy:flashControl & -0.06 & 6.93e-04 & (-0.06, -0.06) & -87.90 & $<$ .001 \\ 
  29 & Red deer &  &  &  &  &        \\ 
  30 & (Intercept) & -6.06 & 0.50 & (-7.05, -5.07) & -12.04 & $<$ .001 \\ 
  31 & time.deploy & -0.06 & 0.14 & (-0.33,  0.21) & -0.44 & 0.658  \\ 
  32 & flash1 & -0.03 & 0.18 & (-0.39,  0.33) & -0.18 & 0.859  \\ 
  33 & flashControl & -0.32 & 0.75 & (-1.80,  1.16) & -0.43 & 0.670  \\ 
  34 & time.deploy:flash1 & 0.37 & 0.19 & ( 0.01,  0.73) & 1.99 & 0.047  \\ 
  35 & time.deploy:flashControl & -0.09 & 0.20 & (-0.48,  0.30) & -0.46 & 0.648  \\ 
  36 & Lynx &  &  &  &  &        \\ 
  37 & (Intercept) & -6.75 & 0.48 & (-7.69, -5.82) & -14.16 & $<$ .001 \\ 
  38 & time.deploy & 0.08 & 0.21 & (-0.33,  0.49) & 0.38 & 0.705  \\ 
  39 & flash1 & 0.39 & 0.29 & (-0.19,  0.96) & 1.32 & 0.187  \\ 
  40 & flashControl & -0.33 & 0.65 & (-1.59,  0.94) & -0.50 & 0.614  \\ 
  41 & time.deploy:flash1 & 0.02 & 0.28 & (-0.54,  0.57) & 0.06 & 0.955  \\ 
  42 & time.deploy:flashControl & -0.23 & 0.37 & (-0.96,  0.50) & -0.62 & 0.538  \\ 
   \hline
\end{tabular}
\end{table}


\caption[Standardised model parameters]%
{Standardised model parameters \par \small Results of generalised linear mixed effect models on detection rate of species at 53 different locations in south-eastern Norway, with three different treatment levels; period with only IR camera (flash:0), period with additional white LED camera (flash:1) and site unchanged through the whole study period (flash:Control). Random effects are location ID and week of year. Standardised parameters were obtained by fitting the model on a standardised version of the dataset. 95\% Confidence Intervals and p-values were computed using the Wald approximation.}

\end{table} % må inn og fjerne end{table} kvar gong tabellen oppdaterest!

 





%\subsection{CPH mixed effect}
%n.obs ~ time.deploy + flash*species + random.eff






