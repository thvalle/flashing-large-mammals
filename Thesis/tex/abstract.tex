\section*{Abstract}

Camera trapping is an increasingly important tool in animal ecology that is generally targeted towards large mammals, and especially large carnivores.
Nonetheless, the cameras are triggered by all large and medium-sized animal species in the area, and thus gathers valuable data on the whole ecological community, like their diel and seasonal activity patterns.
% And
White light flashes are sometimes utilized to get more detailed photos allowing for capture-recapture based population estimates for naturally marked species, like the Eurasian lynx (\textit{Lynx lynx}).
% But
However, the white light could function as a stressor or attractant for different species, which would affect density estimates. 
There is evidence of behavioural change in several mammal species, when exposed to a white flash, but quantifications on the detection rate of species are still lacking.

% In this study
Therefore, I investigated whether introducing an additional white LED camera trap (CT) at established CT sites affected the detection rates of the most common wild mammal species in the area. As CT flashes only are used while ambient lighting is low, I quantified the species' diel patterns in the process.

% I predicted 
I predicted that the detection rate of species with nocturnal and crepuscular activity patterns would be altered as a response to the white light stimuli, and that the extent of the effect would depend on the species' visual acuity. 

% What happened was
The results showed no significant effects of white LED flashes, when compared to IR flashes,
suggesting that white-flash cameras are suitable for studies using indices and capture-mark-recapture estimators. 

\thispagestyle{plain}

%\textit{Keywords:} 
%%Animal behaviour; % Tja, eit klart resultat av åtferd, men eg bruker liten tid på det i oppgåva
%camera trap;
%camera trap shyness; % To the point!
%monitoring bias; % To the point
%night-time photography; % Det er vertfall då det burde ha noko å sei
%diel activity; % Nå som eg har brukt så mykje tid på det
%density estimation; % Hovudgrunnen til å undersøke detection rates


%When in the abstract of an article, authors conclude an effect is “statistically equivalent,” the abstract should also include the equivalence bounds that are used to draw this conclusion.
%TODO Frå TOST-artikkelen til Daniel Lakens (2017)