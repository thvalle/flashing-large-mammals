%The fieldwork was conducted between September 15 and December 20, 2017. In this study, camera traps from the Norwegian Institute for Nature Research (NINA) were used. NINA uses camera traps to monitor the Eurasian lynx in southeastern Norway as part of the SCANDLYNX project (Odden, 2015). SCANDLYNX is a Scandinavian research project on the Eurasian lynx (Odden, 2015; SCANDLYNX, 2017). Camera traps were placed specifically with the goal to photo-capture lynx, and were therefore placed in steep terrain, on ledges or facing the cliff bases, often close to wildlife trails. The cameras were pointing perpendicular to the wildlife trail at locations where a wildlife trail was present. Each camera was mounted on a tree between 0.2 and 1 m above ground

Five different models of RECONYX™ (address: 3828 Creekside Ln, Ste 2, Holmen, WI 54636, USA, www.reconyx.com) cameras were used, 
and one model of BROWNING™ (address: One Browning Place, Morgan, UT 84050, USA, www.browningtrailcameras.com), details in table \ref{tab:cam_mod} and \ref{tab:cam_set}.


\begin{table}[h]
\caption{Camera models included in the survey}
\label{tab:cam_mod}
\centering

\begin{tabular}{llr}
\hline
Producer  & Model name & Flash type  \\
\hline 
Browning	& Spec Ops: Extreme 					& No-glow IR \\
%\multirow{5}{2cm}{Reconyx HyperFire Series} &
			& HC500 Semi-Covert IR					& Red-glow IR \\
Reconyx		& HC600 High-Output Covert IR			& No-glow IR  \\
HyperFire 	& PC800 Professional Semi-Covert IR 	& Red-glow IR \\
Series 		& PC900 Professional Covert IR 			& No-glow IR  \\
    		& PC850 Professional White Flash LED	& White LED  \\
\hline
\end{tabular}
\end{table}


\begin{table}[h]
\caption[Camera settings and features]
{Camera settings and features %\par \small 
All Reconyx-models were part of the HyperFire series and practically identical in all aspects except for type of flash. Camera specifications are gathered from product reviews (www.trailcampro.com).}
\label{tab:cam_set}
\centering
\begin{tabular}{lcc}
\hline 
 & Browning & Reconyx \\ 
\hline 
Number of cameras 	& 34(?) 	& 26(?) \\  
Trigger speed 		& 0.43 s 	& 0.28 s \\ 
Recovery speed 		& 0.8 s 	& 0.9 s \\ 
Photos per trigger 	& 8 		& 3 \\  
Detection angle 	& 45.5$^{\circ}$ 	& 42$^{\circ}$ \\ 
Field of view 		& 40.6$^{\circ}$ 	& 42$^{\circ}$ \\  
Quiet period 		& No delay 	& No delay \\ 
Trigger interval	& Rapid fire & Rapid fire \\
Time lapse			& No	 	& Yes \\
\hline 
\end{tabular} 
\end{table}



%( CT model and settings (quiet period, SENSOR SENSITIVITY, trigger speed, photograph, burst of photographs or video, type of flash, etc.) ) %TODO add into cam_mod table


Cameras were operating 24 hours per day. The RECONYX™ cameras were set to take one time lapse photo per day in order to verify that the cameras had been operational.
They were set to take 3 pictures per series, as fast as possible using \emph{rapidfire}, and retrigger immediately using \emph{no delay}.

The BROWNING™ cameras were also set to rapidfire, but to 8 photos per trigger, which made the memory cards more vulnerable to filling up before being collected. This happened in some areas with sheep and/or cattle, and sometimes due to triggering by vegetation.

Therefore, the BROWNING™ cameras tended to have more gaps of inoperable days, and the number of active camera days are confounded.
To approach the true number of active days, I assumed all BROWNING™ cameras to be functional every day, unless the camera was inactive when I visited it. In that case, I considered the camera inactive since the day of its last photo.




% [Detection shootout 2017](https://cdn.shopify.com/s/files/1/1065/8354/files/2017_Detection_Shootout_8de2600d-eb3a-42a8-9420-728aae5056e5.pdf?12422955367088316008)