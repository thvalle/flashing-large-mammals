Manus  masterforsvar


\section{Intro til presentasjonen} 

In the field of physics the effect of observation has been determined to affect an object itself, as in Einsteins famous elaboration on the nature of a photon. Similarly, in the field of biology, especially zoology, we have been aware of the (rather obvious) effect our observations have on our object of a study. For example, the startling of deer during a count to estimate the population size. Camera traps represent an attempt to by-pass this clear bias in our studies, but now the comparison to an altered photon becomes even clearer. It has long been assumed that camera trapping is a non-invasive study method of terrestrial animals. However, this notion has been contested, and ultimately proven wrong. Even the use of an Infra-red light as a flash can be detectable by many species, not mentioning the auditory cues from the camera shutter, human smell etc, etc (Meek et al. 2014a).
So, the point of this study is to measure the change in the observed photon, so to speak.
