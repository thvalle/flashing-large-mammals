\chapter{Discussion}
%TIPS FRÅ INGVILD: Bare pass på at du svarer på hypotesene dine og drøfter dem

%AM: Gjenta først utfordringen med å telle antall dyr. At det er en rask utvikling av metoder, men at man ikke kjenner effekten av ulike sensorer

%IMR:Det er også fint å starte diskusjonen med et avsnitt som viser litt bakgrunn, som Atle nevner, så kort oppsummert hva du fant, og hvorfor dette er viktige funn/hvordan de passer inn i det større bildet. Så kan man gå videre med underavsnittene der man diskuterer mer detaljer og delmomenter

This study examined the detection rates of nine sympatric mammal species during periods with and without white LED camera traps present. 
Animals can detect CTs using IR flash, both by hearing, smelling and seeing it \autocite{Meek2014a}, but to a varying degree, as their surroundings are filled with distractions. 
However, a white light emitting CT is noticeable for any land dwelling mammal during night, and does startle some individuals \autocite{Meek2014a, Rovero2013, Glen2013a}.
Nevertheless, I found no clear cut evidence that capture rates of any species were significantly impacted by the usage of white LED. 
% Reported on tigers and other species by references in Meek14a

There were examples of individual foxes, roe deer, pine marten and one badger turning around and fleeing, when flashed by a white LED CT, but more examples of species merely observing the CTs.
Most animals seemed indifferent when passing the white LED CTs, and individuals from all species of cervids were observed ruminating for up to several minutes, while being repeatedly flashed.
The latter is also supported by my results. %addition by IMR
Nevertheless, the overall effect of white LED was minimal, suggesting these responses were short term avoidance behaviour and did not lead to longer term avoidance of the sites.
As shown in the density plots (figure 3.2-3.10 a), CTs successfully recorded daily activity patterns for all species, revealing seasonal variation in roe deer, badgers, pine martens, hares and red squirrels. % Diel patterns
Red squirrel was the only diurnal species. Mountain hare and badger showed nocturnal patterns, whereas the rest had crepuscular activity patterns.% Diel patterns remained similar throughout the year, and hence, all species were subject to the white flash at proportional times of day, related 


%%%%%%%%%%%%%%%%%%%%%%%%%%%%%%%%%%%%%%%%%%%%%%%%%%%%%%%%%%%%%%%
\section{The effect of CT flashes on carnivores}

I expected to find detection rates of foxes at least slightly lowered, due to examples of frightened individuals. Surprisingly, the mean detection rates were highest during white LED periods, although the difference was non-significant.
Red fox was the fourth most common species, and so the frightened individuals could represent a small minority that avoided white LED sites. To find out whether the frightened individuals were less likely to be redetected, one would need to recognize individuals. % across-population studies have been over-emphasized and led to a neglect of within-population studies. Difficult in pure CT studies due to small variations in appearance of individuals (Hayes1997)
In a recent study, \textcite{Taggart2020} did just that, studying feral cats' (\textit{Felis catus}) responses to white and IR flash, by using a capture-mark-recapture design.
They found no evidence for white LED CTs affecting redetection, nor that flash type affected behavioural responses to CTs.
Beddari (2019) found lynx and wolves' reactions to CTs to vary with flash type, although she did not quantify the effect on detection rates. Lynx were more dependant on visual cues to detect the camera traps, substantiating their dependence on sight \autocite{Beddari2019}. 
However, just like \textcite{Taggart2020}, I found no effect of white LED on the detection rates of felids. 


Whenever the white LED CTs were absent, an empty metal case remained above the IR CT that often were filled with snails, arthropods and dirt. The empty metal cases can have acted as hiding places and food sources for birds and squirrels. Consequently, they could represent either an attractant or a repellant based on whatever occupied or marked the case.
Could this explain the attractant effect IR periods had on pine marten? If so, my study design confounded the effect of white LED on pine martens detection rates.

Badgers seemed indifferent to the white LED, having almost entirely parallel slopes of detection rates for all three types of periods. 

Although the Second Generation P-Value (SGPV) of the control period's slope were fairly high (82\%), the equivalence test rejected it's practical equivalence to having no effect (more on SGPV below). 


%IMR: Kanskje skrive mer her om hvorfor de har det aktivitetsmønsteret de har, og hvorfor det er likt/ulikt mellom arter? Med referanser til tidligere studier
%DIEL PATTERNS


%TODO konkluderande om carnivores!
Nevertheless, the overall effect of white LED was minimal, suggesting these responses were short term avoidance behaviour and did not lead to longer term avoidance of the sites. %AM: Hence, CT with or without ... are not likely to affect |...This is important for monitoring of lynx




%% Examples - Flytta frå intro %%%%%%%%%%%%%%%%%%%%%%%%%%%%%%%%%%%%%%%







%%%%%%%%%%%%%%%%%%%%%%%%%%%%%%%%%%%%%%%%%%%%%%%%%%%%%%%%
\section{The effect of CT flashes on cervids}
 %AM: litt generelt om hva vi forventer å finne fra andre studier?
%JO: Start å diskutere effekten av blits
\textcite{Henrich2020} studied roe deer and red deer's responses to no-glow IR, red-glow IR and white flash, and found no change in trapping rates over time for any flash type or species. However, they used a xenon white flash, which had a cool down of minimum 22 seconds. The discrepancy in picture frequency between the white xenon and the other two flash types hindered any meaningful comparisons of deer detection rates with white light flashes. 
In my study, cervid detection rates were similar between all periods, and no species showed signs of white LED negatively impacting redetections, supporting the findings of \textcite{Henrich2020}.
Although white LED periods saw a significantly positive trend in red deer detection rates, the difference was non-significant compared to the IR periods from the same sites. As red deer only were present at 26 of the 56 sites, seasonal changes not accounted for by the model random effects could explain the differences between the period types.
\textcite{Henrich2020} found red deer to be more wary of CTs than roe deer in general, supporting the notion that larger animals react stronger than small ones, to human disturbance \autocite{Gaynor2018}.
Although the two larger cervids varied more in detection rates than the smaller roe deer, I suspect these differences to be stochastic variation due to the lower densities of red deer and moose.


% Diel patterns and their influence on the reactions to white LED
Similar to my findings, activity patterns of red deer in Europe have previously been reported as crepuscular, by studies carried out in areas where deer were heavily culled or hunted by humans, and where large carnivores were absent \autocite{Kamler2007}. 
However, \textcite{Kamler2007} found red deer to be cathemeral in the Bialowieza National Park, Poland, where human hunting was prohibited, and healthy populations of both lynx and wolves were present.
As similar activity patterns had been reported on European bison (\textit{Bison bonasus}) in the same forest, \textcite{Kamler2007} suggested ungulates shifts towards nocturnality as a strategy to avoid humans. 
Indeed, the same nocturnal shift has been shown in mammals across diverse mammalian taxa and geographic distribution \autocite{Gaynor2018}.

In my study, most cervids either reacted by passing the white LED CTs unflustered, or by stopping in front of the camera for a minute, inspecting the CTs and possibly scanning the area for other threats. 
Nevertheless, as with the carnivores, the overall effect of white LED was minimal, and did not lead to long term avoidance of the sites. 



%%%%%%%%%%%%%%%%%%%%%%%%%%%%%%%%%%%%%%%%%%%%%%%%%%%%%%%%%%%%
\section{The effect of CT flashes on Glires species}



%Hearing abilites correlates with head size \autocite{Heffner2007}, where species with smaller heads hear higher frequencies than species with larger heads.
%Relevant?
Pine martens are predators of squirrels, and squirrels may try to avoid pine martens.
The negative trend for squirrels during IR periods could be a correlation with the positive trend for pine martens during the same periods.
The picture of the squirrel in figure \ref{ekorn}b, is taken at the same site as the picture of the pine marten in figure \ref{maar}b.

%IMR: Litt mer om fear og avoidance behaviour 


\textcite{Ikeda2016} reported mountain hare (in Japan) as being crepuscular during spring-summer, and nocturnal during autumn-winter, whilst \textcite{Ogurtsov2018} found it to be nocturnal (in Russia). In my study, mountain hares were nocturnal throughout the year, supporting the building evidence on nocturnally shifting mammals in response to human disturbance \autocite{Gaynor2018}. 
%Ikeda was in more secluded forest? He was also at dirt roads...

%Split opp og ta under artsgruppe

%Thus, mountain hare, red fox and pine marten contradicted the expected activity from \textcite{Hall2012}, 

Cont....

%In trying to find a night-time photo taken by a white LED CT of squirrels, none of the photos taken at night time by the IR CTs coincided with a white LED photo. In other words, the few times squirrels have passed the CTs during night, the white LED CT was placed with too high an angle to detect them.
%I did however see two events of a squirrel jolting past the white LED CT at one site during night time. Being mid-air in both photos, it is impossible to infer any behavioural reactions to the experience.




\section{Eye morphology}

This contradicts my assumption that species with high relative acuity would react stronger to a white flash. %Rett etter "just like Taggart2020"


Assuming that cervids are true cathemeral species in their natural surroundings, their eyes would be adapted for a higher acuity than nocturnal species. Relative acuity of cervids calculated with the supplementary data from Hall et al. (2012), supports this notion. Cervids had a medium relative acuity (1.25 - 1.34) compared to the low value for badgers (1.05), and high values for lynx and martens (1.43).

However, I never observed any red deer fleeing from a white LED site, and only observed one young moose running past a CT.
Roe deer expressed flight responses a few times that I saw, and not instantly, but after having been flashed a couple of times. \textcite{Gaynor2018} mentioned some animals being more fearful of predators in lit areas, which could explain the sudden shift in behaviour of some roe deer. The white light might act as "splintering wood", making nearby animals aware of the photo captured individual, and the photo captured individual self aware/more wary of nearby sounds.




In their paper, \textcite{Henrich2020} also reported habituance to novel CTs.
As all the sites in my study were established well before the survey start, animals could have already habituated to their presence. The only novelty presented by additional white LED CTs was my hypothesized flash blindness, and more noise from two jointly triggering CTs.
%The relative acuity of cervids as calculated with the supplementary data from Hall et al (2012), were medium (1.25 - 1.34) compared to the low value for badgers (1.05), and high values for lynx and martens (1.43).

Moreover, most sites were at human and tractor paths, which operates as "high roads" for travelling animals, attracting them to the CT site. Therefore, I am unable to conclude on the effect a white LED would have if put up on a site, and especially if put up at random sites, without any attractant stuff. As always, whether to use a white LED or not depends on the survey goal.
%example of cervids and white flash (+ CTs in general)

%% Eye morphology Flytta frå intro %%%%%%%%%%%%%%%%%%
In the supplementary material, \textcite{Hall2012} provided data on the species they had analysed, with eye measurement data and classification of diel pattern. Three of the species in my study were represented in their dataset, whilst most other were represented by the same genus (except for roe deer), as seen in table \ref{eye}.
\textcite{Hall2012} defined cathemeral species as awake and active both during day and night.

However, in their CT study of seasonal and diel activity patterns, \textcite{Ikeda2016} strictly defined a species as cathemeral when no differences were observed in the photographic frequencies among day, night and twilight.
Since this also is a CT study, I will do the same. Therefore, I expect some contradictions with the species categorized as cathemeral in the supplementary material of \textcite{Hall2012}.

As stated in \textcite{Hall2012}:
"The ratio of corneal diameter to axial length of the eye is a useful measure of relative sensitivity and relative visual acuity that has been used in previous studies as a way to compare animals of disparate size."
Relative acuity is given in table 1.1 as axial length divided by corneal diameter.
The higher the value of relative acuity, the higher the hypothetical importance of sight for each species.
%%%%%%%%%%%%%%%%%%%%%%%%%%%%%%%%%%%%%%%%%%%%%%%%%%%%%%%%%%
\begin{table}
	\centering
	\caption[Data from Hall et al. 2012]
	{\footnotesize Relative acuity from \textcite{Hall2012}. Three of the species in my study were represented in the dataset. The other six have been paired with the closest relative of the dataset.}
	\label{eye}
	\small
	\begin{tabular}{cccc}
	\toprule
	Study & Closest relative in 	& Relative 	& Diel  \\	%relative 
	species& Hall et al. 2012  		& acuity 	& pattern \\	%sensitivity
	\midrule
	Lynx lynx & L. canadensis 		&1.43 	& Cathemeral \\	%& 0.70
	Martes martes & M. flavigula 		&1.43 	& Diurnal 	\\	%& 0.70
	Capreolus capreolus & Dama dama 	&1.34 	& Cathemeral \\	%& 0.74
	Sciurus vulgaris & S. carolinensis	&1.30 	& Diurnal    \\	%& 0.77
	Alces alces & -  			&1.28 	& Cathemeral \\	%& 0.78
	Vulpes vulpes & -		 	&1.26 	& Nocturnal \\	%& 0.79
	Cervus elaphus & C. nippon 		&1.25 	& Cathemeral \\	%& 0.80
	Lepus timidus & L. californicus 	&1.18 	& Cathemeral \\ %& 0.84
	Meles meles & - 			&1.05 	& Nocturnal \\	%& 0.96
	\bottomrule& 
\end{tabular}

\end{table}



%\section{Visual sensitivity}
%%%%%%%%%%%%%%%%%%%%%%%%%%%%%%%%%%%%%%%%%%%%%%%%%%%%%%%%%%%%%%%%%%%%%%%
%The second way is by causing a glare, or flash blindness. 
%In the words of \textcite{Nakagawara2001}: "A typical example of a glare-producing stimulus would be an oncoming automobile’s headlights at night. The visual effects of glare usually cease once the stimulus is removed. However, residual effects, such as spatial disorientation or loss of situation awareness, can persist. 
%Flashblindness is defined as visual loss during and following exposure to a light flash of extremely high intensity (7). An example is the temporary loss or severe reduction of vision experienced after exposure to a camera flashbulb. This type of visual impairment may last for several seconds to a few minutes."
%
%As per \textcite{Dryja2005}, an abrupt increase in illumination will lead to the hyperpolarization of all photoreceptor cells. Moreover, the hyperpolarization of rod cells last longer than that of cone cells. Nocturnal mammals have a higher rod/cone ratio than diurnal mammals.


%A detail to remember is that white LED periods had two active CTs at the same time, which would double the sound stimuli as well. In that sense, I am unable to distinguish whether the animal's responses are due to the visual or the auditory stimuli.
%
%As noted by \textcite{Beddari2019}, species that claim territory will be aware of their surroundings (lynx and wolfs, but also translates to most species in my thesis), which probably makes them conscious about foreign objects.
%
%Badgers and squirrels are reported on expressing the "dear enemy" effect, which is that they react stronger to alien scents, than they react to known scents from neighbouring competitors. Whether this effect translates to scents outside their intraspecies realm, %scents from other things than their own species
%I do not know.
%%%%%%%%%%%%%%%%%%%%%%%%%%%%%%%%%%%%%%%%%


\section{Second Generation P-Values}
The equivalence tests found the detection rate of roe deer, red foxes and badgers to be unaltered during any of the treatment periods.
Those were also the three species with the most events included in the model (grey area of the bars in figure \ref{fig:events}a).
When trimming period lengths before the modelling, many mountain hare events were excluded from the model, and presumably most of them from IR periods. That could explain the negative slope for IR periods in \ref{hare}d, and why the IR slope wasn't accepted as practically equivalent to the control slope.

%Considering the time variable, and its interactions, all are within the ROPE, and the test tells us to accept its practical equivalence to H0, namely that there is no effect. 
When interpreting a continuous variable, like the variable for time since deployment, it is worth considering the variable scale. 
I scaled my time variable to represent 10 day intervals, in order for the model to converge. That means the estimated effect of time since deployment is ten times larger than it would have been if it remained as 1 day intervals.
Conversely, had I scaled it to represent the whole span of 84 days, the estimated effect and confidence interval would have been 8.4 times larger than what it is now, thus leaving all the equivalence tests undecided on the effect sizes of time since deployment.
However, the standard null hypothesis significance test would be unaffected as parameters remain proportional around 0. %log means

%TODO %AM: men hadde det betydning? I så fall hvilken?
Seeing as capture-mark-recapture studies are set to short period lengths

much shorter than mine, it seems reasonable to aim for a time unit which would translate well to these more common lengths.

For example, Henrich et al. (2020) set up cameras for periods of ~20 days, %TODO 20 days true length?
and could therefore not have detected any effects on a time scale of 84 days.
In that sense, a scale of 84 days is meaningless.

%Kutt:
%When looking at the slopes for the detection rate of badger in figure \ref{grevling}, I'm also confident that there were no practical difference between the different types of period.



\chapter{Conclusion}
%JO: Her kan du komme tilbake til utgangspunktet  
%Noe sånt?: 
Camera trapping is an increasingly important tool in animal ecology and wildlife conservation. The underlying assumption for using CT in investigating multiple species are that CTs are unselective in which species they capture, or that biases in capture rates can be corrected for. An accurate interpretation of data from camera trap studies is dependent on understanding of how study design decisions such as the flash type may influence the trapping rates of the target animals.  I found no evidence for that capture rates of any of the nine mammal species were significantly impacted by the usage of white LED. My findings suggest that white-flash cameras are suitable for capture-mark-recapture studies, and quantification of diel patterns. 
It is important to note that the probability of detecting a species with a camera trap is affected by several other factors operating on different scales ..

% JO flytta frå diskusjon intro
%Monitoring population densities and activity patterns of mammal species are challenging, as many of them are nocturnal and elusive. 
%The usual approaches are variants of direct observations and telemetry surveys, but they can be costly, invasive and can be prone to undercounting \autocite{morellet2011, Ikeda2016}.
%Camera traps are quickly becoming a widely used survey method, because of low costs, low-invasiveness and reliable, standardised data \autocite{Burton2015}.




%%%% Viktige poeng henta frå andre studier %%%%%%%%%%%
	%In some studies the target species’ ability to detect a camera trap may not be important because the requirement is to detect presence only, so irrespective of whether the animal baulks and runs from a camera trap is of no importance (Meek2014a conclusion)



%%%%% Eit arkiv %%%%%%%%%%%%%%%%%%%%%%%%%%%%%%%%%%%%%%
%AM: Dette ville jeg strøket
	
	


