\chapter{Discussion}
%TIPS FRÅ INGVILD: Bare pass på at du svarer på hypotesene dine og drøfter dem

This study examined the detection rates of nine sympatric mammal species during periods with and without white LED camera traps present, and their activity patterns. 
Animals can detect CTs using IR flash, both by hearing, smelling and seeing it \autocite{Meek2014a}, but to a varying degree, as their surroundings are filled with distractions. 
However, a white light emitting CT is noticeable for any land dwelling mammal during night, and does startle some individuals \autocite{Meek2014a, Rovero2013, Glen2013a}.
Contrary to my prediction, I found no clear evidence that capture rates of any species were significantly impacted by the usage of white LED. 
% Reported on tigers and other species by references in Meek14a

There were examples of individual foxes, roe deer, pine marten and one badger turning around and fleeing, when flashed by a white LED CT, but more common were examples of species merely observing the CTs.
Most animals seemed indifferent when passing the white LED CTs, and individuals from all species of cervids were observed ruminating for several minutes, while being repeatedly flashed.
Nevertheless, the overall effect of white LED was minimal, suggesting these responses were short term avoidance behaviour and did not lead to longer term avoidance of the sites.
As shown in the density plots (figure 3.2-3.10 a), CTs successfully recorded diel activity patterns for all species, revealing seasonal variation in roe deer, badgers, pine martens, hares and red squirrels. % Diel patterns
Red squirrel was the only diurnal species. Mountain hare and badger showed nocturnal patterns, whereas the rest had crepuscular activity patterns.% Diel patterns remained similar throughout the year, and hence, all species were subject to the white flash at proportional times of day, related 


%%%%%%%%%%%%%%%%%%%%%%%%%%%%%%%%%%%%%%%%%%%%%%%%%%%%%%%%%%%%%%%
\section{Activity patterns and the effect of CT flashes}

\subsection{Carnivores}

I did not detect an effect of white LED on badgers and red foxes.
Lynx and pine martens showed lower detection rates during control periods, than during IR and white LED periods.
Due to examples of frightened individuals, I expected to find detection rates of foxes at least slightly lowered. Surprisingly, the mean detection rates were highest during white LED periods, although the difference was non-significant.
Red fox was the fourth most common species, and so the frightened individuals could represent a small minority that avoided white LED sites. To find out whether frightened individuals were less likely to be redetected, one would need to recognize individuals. % across-population studies have been over-emphasized and led to a neglect of within-population studies. Difficult in pure CT studies due to small variations in appearance of individuals (Hayes1997)

In a recent study, \textcite{Taggart2020} did just that, studying feral cats' (\textit{Felis catus}) responses to white and IR flash, by using a capture-mark-recapture design.
They found no evidence for white LED CTs affecting redetection, nor that flash type affected behavioural responses to CTs.
Beddari (2019) found lynx and wolves' reactions to CTs to vary with flash type, although she did not quantify the effect on detection rates. Lynx were more dependent on visual cues to detect the camera traps, substantiating their dependence on sight \autocite{Beddari2019}. 
However, just like \textcite{Taggart2020}, I found no effect of white LED on the detection rates of felids. 
Whenever the white LED CTs were absent, an empty metal case remained above the IR CT that often got filled with snails, arthropods and dirt. The empty metal cases can have acted as hiding places and food sources for birds and squirrels. Consequently, they could represent either an attractant or a repellant based on whatever occupied or marked the case.
Could this explain the attractant effect IR periods had on pine marten? If so, my study design confounded the effect of white LED on pine martens detection rates.

Nevertheless, the overall effect of white LED was minimal, suggesting these responses were short term avoidance behaviour and did not lead to longer term avoidance of the sites.  
Hence, CT with or without white flashes are not likely to affect the four carnivores in this study.
This is important for the monitoring of lynx as white light flashes provide detailed photos which can be used to distinguish between individuals through their coat patterns.
Being able to recognize individuals allows for capture-mark-recapture study designs \autocite{Rovero2013}, and higher accuracy of species identification \autocite{Glen2013a}.


The daily activity patterns of badgers remained identical throughout my study, although the overall activity level varied, seeing an especially large increase during spring. 
\textcite{LauraCab2020} found the activity patterns of badgers to be more affected by temperature and time of day, than photoperiod, which hints at a low importance of visual senses. The peak in activity during spring was also reported from Russia \autocite{Ogurtsov2018}, and is likely due to food availability (earth worms) and the breeding season \autocite{LauraCab2020}.
Lynx were reported as crepuscular to cathemeral in the Russian study on diel patterns \autocite{Ogurtsov2018}, which argued that access to prey was the main cause for lynx diel patterns, in favour of ambient light. They also noted that lynx elicit a cathemeral activity pattern in areas protected from human disturbance.
Foxes have been reported on having similar activity patterns throughout the year \autocite{Ikeda2016}, whilst pine martens vary from low and nocturnal activity in the winter, towards cathemeral patterns in the summer breeding season \autocite{Zalewski2000}, both supported by my findings.
Red foxes were found to be more diurnal in areas with low human impact, whereas Zalewski's (2020) report on pine martens were from the Bialowieza National Park were human impact is low, further substantiating the claim that smaller animals react less to human disturbance, than large animals \autocite{Gaynor2018}.
Conclusively, using a subset of camera traps deployed by NINA, I was able to find diel and seasonal patterns of four carnivores coinciding with earlier findings in the literature.

%%%%%%%%%%%%%%%%%%%%%%%%%%%%%%%%%%%%%%%%%%%%%%%%%%%%%%%%
\subsection{Cervids}

Contrary to my prediction, cervid detection rates were similar between all periods, and no species showed signs of white LED negatively impacting redetections.
\textcite{Henrich2020} studied roe deer and red deer's responses to no-glow IR, red-glow IR and white flash, and found no change in trapping rates over time for any flash type or species. However, they used a xenon white flash, which had a cool down of minimum 22 seconds, effectively hindering any meaningful comparisons of white flash detection rates with the other two flash types. 
Although white LED periods saw a significantly positive trend in red deer detection rates, the difference was non-significant compared to the IR periods from the same sites. As red deer only were present at 26 of the 56 sites, seasonal changes not accounted for by the model random effects could explain the differences between the period types.
In my study, most cervids either reacted by passing the white LED CTs unflustered, or by stopping in front of the camera for a minute, inspecting the CTs and possibly scanning the area for other threats. 
As with the carnivores, the overall effect of white LED was minimal, and did not lead to long term avoidance of the sites. 
% Kutter like gjerne dette:
%\textcite{Henrich2020} found red deer to be more wary of CTs than roe deer in general, supporting the notion that larger animals react stronger to human disturbance, than small animals \autocite{Gaynor2018}.
%Although the two larger cervids varied more in detection rates than the smaller roe deer, I suspect these differences to be stochastic variation due to the lower densities of red deer and moose, .

% Diel patterns and their influence on the reactions to white LED
Similar to my findings, earlier telemetry studies have shown crepuscular activity patterns for roe deer, moose \autocite{Cederlund1989} and red deer \autocite{Godvik2009}.
However, \textcite{Kamler2007} found red deer to be cathemeral in the Bialowieza National Park, Poland, where human hunting was prohibited, and abundant populations of both lynx and wolves were present. 
The activity patterns of ungulates seem to be driven by similar limitations in forage resources and avoidance of human disturbance \autocite{Cederlund1989, Kamler2007}. 
My findings match the expected findings of higher activity in summer (spent foraging for easily digestible plant material), and lower activity during winters (spent ruminating on lower quality plant material).

%%%%%%%%%%%%%%%%%%%%%%%%%%%%%%%%%%%%%%%%%%%%%%%%%%%%%%%%%%%%
\subsection{Glires}

Neither squirrel nor mountain hare detection rates were significantly affected by white LED.
However, squirrel detection rates had a significantly negative trend compared to the control periods. Pine martens are predators of squirrels \autocite{Zalewski2000}, and squirrels may therefore try to avoid pine martens. The negative trend for squirrels during IR periods could be correlated with the positive trend for pine martens during the same periods. 
Many mountain hare events were excluded from the model when I trimmed the period lengths, and presumably most of them from IR periods. That could explain the negative slope for IR periods in figure \ref{hare}d, and why the IR slope wasn't accepted as practically equivalent to the control slope.
Moreover, it is also worth considering the scaling when interpreting effect sizes of continuous variables, like the variable for time since deployment. 
I scaled my time variable to represent 10 day intervals, in order for the model to converge. Consequently, the estimated effect of time since deployment was ten times larger than it would have been unscaled, as one day intervals.
Conversely, had I scaled it to represent the whole span of 84 days, the estimated effect and confidence interval would have been 8.4 times larger than what it is now, thus leaving the equivalence tests of all species undecided on the effect sizes of time since deployment.
Notwithstanding, the standard null hypothesis significance tests were unaffected as parameters remain proportionally distributed around 0. %log means

Red squirrels are clearly adapted for diurnal activity, which has been observed regardless of potential predators being present \autocite{Ikeda2016}. I found spring to be least active time of year for red squirrels.
Both \textcite{Ikeda2016} and \textcite{Ogurtsov2018} reported mountain hares as being nocturnal during autumn-winter and crepuscular during spring-summer. Their high activity in spring was explained by molting and breeding season.
I also found mountain hares to be more active during spring, but they were nocturnal throughout the year, supporting the building evidence on nocturnally shifting mammals in response to human disturbance \autocite{Gaynor2018}. 

\section{Eye physiology and white flashes}

Eye morphology in animals differ with diel activity patterns, e.g. between nocturnal and diurnal species \autocite{Schmitz2010}. Most mammals vary less in eye morphology than other amniotes (birds and reptiles) \autocite{Schmitz2010, Hall2012}, but they have other adaptations to increase light sensitivity. Nocturnal mammals have a higher rod cell to cone cell proportion in their retina, than diurnal mammals, sacrifizing colour vision and visual acuity for higher light sensitivity (Solovei et al. 2009). Moreover, their rod cells are more efficient \autocite{Solovei2009}, and they have the reflecting intraocular structure, tapetum lucidum, which acts as a light amplifyer and causes the 'eye-shine' seen in night photographs \autocite{Ollivier2004}.
Eyes more light sensitive than the human eyes could react stronger to white flashes \autocite{Dryja2005}, and hence, I expected white LED CTs to pose an additional stressor on night-active species in the form of flash blindness \autocite{Nakagawara2001}. However, most mammals rely less on optical senses than humans, and so I used relative visual acuity as a measure of hypothetical importance of sight for each species.

In their supplementary material, \textcite{Hall2012} provided data on the species they had analysed, with eye measurement data. Three of the species in my study were represented in their dataset, whilst most other were represented by the same genus (except for roe deer). Relative visual acuity is given as axial length divided by corneal diameter.
Of the species in my study, lynx and pine martens ranked the highest (1.43), cervids, squirrels and red foxes ranked medium (1.25 - 1.34), whilst mountain hares (1.18) and badgers (1.05) ranked lowest. 
Contrary to my predictions, I found no evidence supporting that visual acuity influence the impact of white LED on detection rates. 
An argument for flash blindness could still be made when using xenon flashes, as they are stronger than white LEDs \autocite{Rovero2013}.
To quantify the effect, the researchers would need to deploy a second CT with the ability to take consecutive photos or shoot video, as have already been proposed \autocite{Glen2013a, Henrich2020}.


Henrich et al. (2020) found evidence suggesting habituation to CTs, arguing that short disturbances not connected to dangerous situations did not lead to long-term avoidance.
The white LED CTs in my study, were used as additional flashes at already established IR CT sites. 
The white LEDs offered a novelty in emitting white light and additional sound, but this was not enough to significantly alter the trapping rates to any of the nine species.
In other words, habituated animals were unaffected by the increase of light and sound emittance in my survey.

There are growing evidence on CTs being low-invasive rather than non-invasive \autocite{Meek2014a,Meek2016,Beddari2019,Henrich2020}, but our expectation of white light affecting animals more than IR CTs is still largely based on anecdotal evidence and expectations.
I argue that the expected difference between IR and white light cameras is largely due to confirmation bias from CT operators and surveyors
Seeing is believing, and so, the conspicuous white light has mislead us to give more weight to examples of white flashes affecting animals negatively.
In reality, unsuspecting animals will be surprised regardless of the sound or light wavelengths being emitted, as long as they are detectable. 



\chapter{Conclusion}
%JO: Her kan du komme tilbake til utgangspunktet  
%Noe sånt?: 
Camera trapping is an increasingly important tool in animal ecology and wildlife conservation, exemplified by my findings on activity patterns. Using a subset of camera traps deployed by NINA, I was able to find diel and seasonal patterns of nine sympatric mammal species, which matched earlier findings in the literature. Mountain hares, cervids, lynx and red foxes showed signs of nocturnal shifts due to human disturbance, whilst badgers, pine martens and red squirrels were unaffected.

The underlying assumptions for using CTs to investigate multiple species are that CTs are unselective in which species they capture, or that biases in capture rates can be corrected for.
An accurate interpretation of data from camera trap studies is dependent on understanding of how study design decisions such as the flash type may influence the trapping rates of the target animals. 
I found no evidence that capture rates of any of the nine mammal species in my study were significantly impacted by the usage of white LED. My findings suggest that white-flash cameras are suitable for studies using indices and capture-mark-recapture estimators. 
It is still uncertain if some frightened individuals are less likely to be redetected, a question that can only be answered by surveying marked individuals responses to both IR and white light flashes.
There are several reports on startled animals reacting to the sound output of CTs before the white flash went off. 
As can be seen, nocturnally active mammals rely on all their sensory inputs when interpreting their surroundings, highlighting the many ways we can, and do, disturb our target species when monitoring them.


%It is important to note that the probability of detecting a species with a camera trap is affected by several other factors operating on different scales .. %TODO !!!!!!!!


%%%% Viktige poeng henta frå andre studier %%%%%%%%%%%
	%In some studies the target species’ ability to detect a camera trap may not be important because the requirement is to detect presence only, so irrespective of whether the animal baulks and runs from a camera trap is of no importance (Meek2014a conclusion)



	
	


