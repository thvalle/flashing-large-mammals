
Discussion

Animals can detect CTs using IR flash, both by hearing, smelling and seeing it (Meek2014a), but to a varying degree, as their surroundings are filled with "distractions". However, a white light emitting CT is noticeable for any land dwelling mammal during night, and do startle some individuals (Burton(?), Meek, Hofmeester and refs therein). 
Indeed, I've seen several examples of individuals turning around and fleeing from whence they came on "my own" pictures. However, I've seen many more examples of animals staying calm and collected in front of white LED CTs, and individuals from all species of cervids were observed ruminating in front of a white LED CT going off repeatedly for several minutes.


_____________________________
Expected result

Considering that this hasn't been studied much before, it is clear that a strong effect on most species was not to be expected.
Grey wolfs in Southeastern Norway have shown a clear aversion to the white light, and as such, NINA has stopped using white flash CTs in areas where they expect to find wolfs (John Odden pers.comm.). In other words, strong effects are quickly noticed, and acted upon from scientists.

As some individuals of many species react, however, the hypothesis has been that for some species, parts of the population could react to a white flash the same way, and thus the CT site would become a sort of an "area of fear"(landscape of fear), reducing the amount of (mainly) redetections (seeing as most animals only would experience the flash when triggering it themselves).
Some individuals could also experience the flash from a distance, if the CT was triggered by vegetation or another animal (pack/group animals, prey flashed when predator close), but these would be rare events.

_____________________________
What does it mean considering theory and/or other studies?

-

_____________________________
Can you generalize?

-
-

_____________________________
Compare to other studies

-
-

_____________________________
Alternative interpretations?

-
-

_________________________________
Strong and weak sides of my study
/ Discussing method

-Study design
	- 60 sites vs 40 (exclude ctrl)
	   -shorter field work
	   -LED-moving quicker
	- Include data from previous year?
-Analysis
   -Scaling of Time variable
	-increased estimate and SE
	-H0 undecided
   -More random-effects?
	-CT height, habitat etc
   -Habitat * flash?
	 -roads and large trails work as an attractant/funnel for travelling animals. 
	   -maybe a wLED stressor is too small to counteract the funneling effect
		-thus only CTs at animal paths would significantly affect detection rates
_____________________________
Implications for practice?

-as have been noted by many; no wLED for behaviour studies
	-(stating the obvious, as we want to be a "fly on the wall" in atferd-studies
-If anything, beware attractant effects first week of LED (ref. surv-plots)
-No implication for occupancy (presence\absence)
-Possible implication for SCaptRecapt(SCR)
	xx-Araujo found that female jaguars avoid(/or seldom use) areas were CTs were put up, compared to male jaguars. Could the same be true for lynx? 


____________________________
More research needed?

-Some times a species can avoid the lens
  -for me in particular, the LED could've scared individuals off even before they entered the IR frame (Browning trigger!).
     -other studies could try to couple a lagging(!) LED-flash to an IR(/black) CT
	-thus allowing for a detection before the white light stimuli
-SCR setup with multiple CT flash-types, in 1-2 home range scale, random placement
	-noting whether individuals avoid the white flash CTs


____________________________
Give recommendations (for practice)

-
-