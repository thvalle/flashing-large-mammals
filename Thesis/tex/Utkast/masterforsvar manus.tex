Manus råmateriale  masterforsvar


\section{Intro til presentasjonen} 

In the field of physics the effect of observation has been determined to affect an object itself, as in Einsteins famous elaboration on the nature of a photon. Similarly, in the field of biology, especially zoology, we have been aware of the (rather obvious) effect our observations have on our object of a study. For example, the startling of deer during a count to estimate the population size. Camera traps represent an attempt to by-pass this clear bias in our studies, but now the comparison to an altered photon becomes even clearer. It has long been assumed that camera trapping is a non-invasive study method of terrestrial animals. However, this notion has been contested, and ultimately proven wrong. Even the use of an Infra-red light as a flash can be detectable by many species, not mentioning the auditory cues from the camera shutter, human smell etc, etc (Meek et al. 2014a).
So, the point of this study is to measure the change in the observed photon, so to speak.





\section{Fri flyt etter ein dusj 05.02.2021} 

\subsection{Introduksjon} 

Eg har slitt med å forstå kva som blir presentert på alle masterforsvar eg har sett på så langt. Derfor vil eg benytta denna anledningen til å formidla oppgåven min på ein måte som mor mi kan forstå. Eg har allereie formidla den meir faglig og traust i skriftform, og får lov til å diskutere den ytterligere med sensorene etter presentasjonen. Så eg forbeholder meg retten til å bruke eit forståelig og enkelt språk akkurat nå.

Mitt prosjekt har gått ut på store pattedyr. Mennesker har alltid prøvd å forstå verden rundt seg, og store pattedyr har for monge vore ein sterk fascinasjon og eit matfat -  og er det idag også (bilde av huleteikninger e.l?).
For meg er ville pattedyr stort sett fascinasjon, og har alltid vært det. Spesielt kattedyr. Så motivasjonen min for å begi meg ut på dette prosjektet var gaupa!
Eg tok kontakt med NINA angåande scandlynx-prosjektet deira, og fekk raskt positivt svar. Så møtte eg John Odden, ein som også var fascinert av gaupa, og som blei veilederen min. Me diskuterte mulige prosjekt. Alle alternativene var knytta til viltkamera.

%Forkast eller skriv kortere:
%Så sikra me oss Atle Mysterud som veileder "på huset" som det kalles, og sidan han har mest erfaring med veiledning blei han hovudveilederen min. Maren Rivrud var eit perfekt bindeledd som teknisk/dataveileder sidan hu hadde vært PhD under Atle, og nå skulle gå over i NINA. Til slutt blei Neri - PhD på NINA - supplert inn som ekstra datakyndig veileder, og det er han eg har belaga meg på mest til slutt.
%Eit spørsmål skilte seg ut som interessant, litt fordi det handla litt om gaupe - og fordi eg fekk dra ut i felt og samle data sjølv. Spørsmålet var kor vidt måten me studerer dyrene påvirker dyrene i seg sjølv.

Å studere pattedyr er nemlig notorisk vanskelig. Dei er stort sett nattaktive, og er også utrusta med eit litt sterkere sanseapparat for nattens mulm og mørke - enn oss sjølv. Dei fleste tidlegare studieteknikkene har enten vore for dyre, eller for invasive. For eksempel gaupa har blitt flydd etter med helikopter for å få studert dei. Det skremmer dyra - og det koster penger. Så har dei blitt bedøvd for at forskerne skal få feste GPS-sporere på dei og at ein veterinær kan ta ulike prøver og sjekke helsa. Dette er definitivt også dyrt, og invasivt - gaupa er ikkje akkurat kjøyredyktig etter ein slik dose.
Og til slutt får man mykje informasjon - om eit individ.

I tillegg dukker det opp ein litt filosofisk bit som også Einstein oppdaga i sin tid. For han handla det om fotoner. Og det at fotoner er forskjellige utifrå om dei blir observert eller ikkje.
For fotoner er det slik at dei er bølger om dei får vere i fred, men om me ønskjer å observere dei er dei i staden partikler. \emph{Eller framstår som partikler!}

No er ikkje eg fysiker, så ikkje ta mitt ord på det. Men det same er sant for oss som skal studere pattedyr. Ei einsam gaupe på fjellet oppfører seg annerledes enn ei gaupe på fjellet - som blir jakta med helikopter!

Kor pålitelig er informasjonen me sitter igjen med etterpå?

Som nevnt er altså mitt prosjekt om viltkamera. Dei har blitt betre og billigere med åra, og er derfor viden brukt innan vitskapen om fritt vilt. Det gir oss høve til å innhente informasjon om ville dyr på ein mindre invasiv måte, og i større omfang. Nå får me plutseleg informasjon om 50 gauper, ikkje berre 1, og me får den utan å skremme dyra så sterkt - og me får den på ein standardisert måte. Altså at samanlikna med to personer som prøver å telle rådyr, så vil kameraene sin telling av dei same rådyrene likne meir på kvarandre.

Det gir oss potensielt litt meir pålitelig data.

Men, framleis har me problemet med at dyr som blir observert er annerledes enn dyr som går i fred. %Referanse til "Are we getting the full picture", Meek etal ?

Dyra kan både se, høyre og lukte viltkamera, til tross for bruk av Infraraud blits. Det har blitt utvikla svart blits som skal vere endå mindre synlig for dei fleste dyr, og det ser ut til å ha ein god effekt (Gibeau ML 2009?). Men bildene blir ikkje så bra da.

Du ser at dette er ein gaupe, vertfall vis du har sett monge slike bilder av gauper. Men du ser ikkje mønsteret på pelsen hans. Så du kan ikkje identifisere han, slik man gjerne vil gjere på slike arter.

Så nokon har sjølvsagt funnet ut at - drit i om dei skjønner at me ser på - eg vil ha gode bilder! Og bruker blits.

Det er heftig å bli utsatt for mitt ute i skogen! Eg har sjølv opplevd det, og ein blir heilt blinda. Du veit sjølv korleis det er når nokon skrur på lyset etter at du har sett ein film i eit mørkt rom.



Så då blir sjølvsagt spørsmålet:  Påverker denne blitsen dyra?

I så fall burde me ta hensyn til det når me tolker dataene våre. Me ønskjer jo å finne ut korleis dyr er til vanleg.
Me ønskjer for eksempel å kunne anslå kor monge hjort som befinner i Nordmarka. Det blir feil anslått om hjorten unngår treet han veit det står eit blinkande kvitt lys på. Då ser me han ikkje lenger.

Derfor har eg stilt spørsmålet: Flashing Large Mammals - Do they mind?


For å finne det ut må eg komme med nokre påstander som kan testast.

Nullhypotesen er så klart: Nei, dei bryr seg ikkje. Kamera med kvit LED blits fanger like monge dyr som kamera utan.

Den alternative hypotesen min er: Jo, enkelte artar bryr seg, og dei bryr seg truleg i forskjellig grad. Ein grevling, som knapt nok ser likevel, bryr seg kanskje lite. Mens ein rev, som ser godt, bryr seg kanskje mykje.

I tillegg lurer eg på: Bryr urbane dyr seg mindre? Altså, dyr som er utsatt ofte for kunstig lys om natta (ALAN), blir dei mindre skremt eller stressa av ein kvit blits, slik at ein Oslo-rev gjerne kjem tilbake, mens ein rev frå Flå blir skremt av slikt teknisk vrøvl?


\subsection{Metode}

For å gjere det fekk eg tilgang til 60 kamera frå Oslo og Viken. Enkelte kamera blei utelatt pga problemer, så eg sto igjen med ~56 til slutt.
40 kamera fekk blits i tre månaders perioder, som veksla mellom seg - 20 og 20. 20 kamera forblei som dei var.

På dei 40 kameraene med blits skrudde eg opp eit ekstra kamera over. Boksene fikk henge, mens kamera blei bytta ut.


Bildene frå kvart infrarøde kamera blei sortert av dei kjekke folka på NINA. Først kjøyrte ein data gjennom bildene for å gjette kva for slags dyr som var der - og foreslå arten med eit tal på kor sikker den var. Deretter gjekk eit menneske gjennom bildene for å sjå om den hadde rett, og retta opp feilene.
Det er ekstra nyttig fordi av og til ser dataen eit ekorn som menneske kanskje ikkje hadde oppdaga!
Og fordi i framtida blir kanskje maskinen god nok til å fjerne menneskebilder sjølv. Og då blir datatilsynet fornøyd med datasikkerheten, og NINA kan få lov til å samle inn meir data om korleis dyr som grevlingen og reven benytter seg av byen på natta!


Eg satt igjen med eit excel-ark som inneholdt informasjon om alle bildene som blei tatt. XXX var "nothing" -korav XXX var timelapse, XXX var antakeleg greiner som blåste i vinden og XXX var kanskje eit dyr som sprang så fort at kamera ikkje rakk å fange det opp.

I tillegg fekk eg disse artene: Mennesker og kjøretøy er ikkje interessant. Fugl er for generell gruppe. Sau og ku er heller ikkje interessant i denne samanhengen. Ekorn og harer er for små for disse kameraene til at dei vil bli fanga opp forutsigbart. Kameraene blei som tidlegare nevnt hengt opp med tanke på gaupe! Så det er ikkje pålitelig at kvart kamera fanger opp alle ekorn og harer som løper forbi like ofte.

Til slutt har eg 6 arter som dukka opp ofte nok på kameraene mine.


For å sjå om dei blei påvirka av blitsen skal eg sjå om dei dukker opp like ofte i perioder med blits som i perioder utan blits. Faktisk skal eg vere litt flinkere enn det. Eg brukte tidspunktene som LED-kameraet gjekk av for å finne ut om eit dyr blei flashed eller ikkje, og bruker vidare den informasjonen for å sei om det påvirka dyra.


For å gjere det må eg kunne sei at det var signifikant. Trillar eg ein terning monge nok gongar får eg kanskje seks monge gonger på rad, og det same kan man sei om kor ofte eg får ein art på eit kamera.

Skal eg kunne motstride nullhypotesen om at ein art ikkje bryr seg om det kvite lyset, må eg kunne vise at det er mykje nok data som tilseier at det er sannsynlig.


Eg brukte GLMM osvosv



\section{29.03 - The world is changing}
If we had no old collection of plants, and wanted to discover their amount of stomata openings in their leaves, we would conclude that plants have a lower number of stomata openings in general than what was true 50 years ago.
The difference between stomata openings in plants and behaviour in animals, is that we cannot verify the latter at this point, using new technology. We only have the word of our predecessors to rely on.
Thus, we can't know for sure if they were right, seeing as they had less reliable ways to conduct diel pattern surveys.
However, we can still investigate animals living in \textit{wilder} surroundings, and use that as a window into the past.





