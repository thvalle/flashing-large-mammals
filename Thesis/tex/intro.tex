\chapter{Introduction}
%Usually, the lynx has been counted using snow tracks in the winter (Odden 2015). However, lately there has been a variable lenght of the snow season, which has made the snow track counts unpredictable and difficult to conduct at a consistent time of year (Odden 2015).
%
Estimating number of animals are central in population ecology, and census methods has always been under development in order to get accurate, reliable ways of conducting surveys (REF!). %TODO
Counting mammals directly is often difficult, as many species are shy, elusive, and mainly night active.
Some methods involve walking one or several persons in a row, scaring up animals, and counting them as they flee.
Still, any such method is prone to undercounting, due to low visibility in dense forests and lack of focus from observers after a while.
There are other, indirect methods, like counting tracks or calls during the mating season, but these work best when the target species has a large density. %AMkomm: hele avsnittet må få på referanser

In northern areas, like Norway, counting animal tracks in snow has been a popular method. %TODO ref!
Snow track counts has the advantage of highly visible tracks, and a somewhat accurate dating of the tracks to the last snowfall, as old tracks vanish.
However, lately the snow season in southern Norway has been variable, which makes snow track counts unpredictable and difficult to conduct at a consistent time of year (\cite{Odden2015}).

Therefore, the Norwegian Institute of Nature Research (NINA) has started to use camera traps (CTs) as a substitute method to monitor family groups of Eurasian lynx (\textit{Lynx lynx}) in southereastern Norway  (\cite{Odden2015}). The surveys are integrated in a coordinated Scandinavian science project on lynx, called Scandlynx. % Beddari siterte SCANDLYNX2017  

CTs have been developing fast, and become quite affordable (\cite{Burton2015}).
They offer a consistent, standardised sampling method which also records date, hour and, in some instances, temperature.
CTs normally use infra-red (IR) light to photo-capture animals, which is invisible to the human eye, but has been shown to be visible to several other mammals (\cite{Meek2014}).
However, the lack of sharpness and detail from IR photos limit the information we can retrieve from them, e.g. individual variation in coat patterns which can be used in capture-recapture models to accurately estimate population numbers. 

The need for more detail has led to the usage of white light flashes.
CTs with white light flash comes with either white xenon or white light-emitting diode (LED) technology, where xenon provides the sharpest photos, but has the disadvantage of requiring long cool downs after each photo (minimum of 22 seconds %TODO 
in \cite{Henrich2020}).

Naturally, white light is highly visible to all land dwelling mammals, and will likely affect the animals to some extent (e.g. startle, stress) which in turn could bias the data we collect. Problems related to CT awareness and behavioural changes have already been discussed by many (\cite{Meek2014}, \cite{Burton2015}, \cite{Hofmeester2019}). %TODO read their citings and replace for papers that actually discuss this theme, not just refers to it.
Beddari (2019) showed that grey wolfs (\textit{Canis lupus}) in Norway tend to shy away from sites where a white LED CT was used, whilst the  lynx  seemed less bothered. %AMkomm: not to be affected? how was this done?
Henrich (2020) studied roe deer (\textit{Capreolus capreolus}) and red deer's (\textit{Cervus elaphus}) responses to IR, black and white flash, but they used a xenon white flash, which has a long cool down, and hindered any meaningful comparisons of deer detection rates with the other flash types.


%***

%Litt om estimeringsmetoder og problemer - og at det er særlig vanskelig for nattaktive og sky arter.
%Litt om bruk av kamerafeller og metodikk som er i enorm utvikling. Binde opp til NINA arbeid og samarbeid i Skandinavia

%
%Capture Recapture models are only available for naturally marked species (e.g. tigers \textit{Panthera tigris}, leopards \textit{Panthera pardus}). 
%Nevertheless, the majority of wildlife species are not easily individually identifiable from photos, rendering CR approaches difficult and leading to widespread interest in alternate analytical approaches for ‘unmarked’ species" \cite{Burton2015} %Dette vil eg ha inn i samanheng med årsaken til LED-bruk



In this study, I will quantify how the usage of white LED flash affects the detection rate of the most common large mammal species in an area in Southern Norway.
I have restricted the analysis to all wild species observed at least 50 independent times during my survey, which totaled nine species.
Namely roe deer, red fox (\textit{Vulpes vulpes}), badger (\textit{Meles meles}), moose (\textit{Alces alces}), red deer, red squirrel (\textit{Sqiurus vulgaris}), hare (\textit{Lepus timidus}), European pine marten (\textit{Martes martes}) and lynx. 

White LED CTs have similar recovery speeds to that of regular IR CTs, as both utilize LEDs as flashes, which makes them well fit for meaningful comparisons.

\subparagraph{Hypotheses} 
%AMkomm: Du kunne selvsagt også tatt med noe om biologi her. Hva du forventer hvis CT gir et godt bilde av aktiviteten

(H0): Usage of white LED flash will have no effect on the detection rate of any species.

(HA): Usage of white LED flash will stress one or more species in general, and therefore lower the detection rate of the stressed species. The effect will likely vary in extent between species. %AMkomm: Vær spesifikk     I expect a stronger effect on night-active than day-active species - (ha arter i parantes)


%* Hypothesis 2: The effect of the white LED will correlate with urbanisation-factors, as individuals that live closer to urban areas are habituated to Artificial Light At Night (ALAN), and thus will have a weaker response to the white LED



