\chapter{Introduction}


How does the usage of white LED flash affect our data?
%AMkomm:Jeg ville startet med behovet for å estimere antall dyr. 
%At det er selve grunnlaget for bestandsøkologien. 
%Få med at det er utfordrende, fordi dyrene ikke kan observeres lett i tett skog.



%Litt om estimeringsmetoder og problemer - og at det er særlig vanskelig for nattaktive og sky arter.
%Litt om bruk av kamerafeller og metodikk som er i enorm utvikling. Binde opp til NINA arbeid og samarbeid i Skandinavia


Capture Recapture models only available for naturally marked species (e.g. tigers \textit{Panthera tigris}, leopards \textit{Panthera pardus}). 
"Nevertheless, the majority of wildlife species are not easily individually identifiable from photos, rendering CR approaches difficult and leading to widespread interest in alternate analytical approaches for ‘unmarked’ species" \cite{Burton2015} %Dette vil eg ha inn i samanheng med årsaken til LED-bruk


Camera traps (CTs) give us the opportunity to monitor in a quantifiable, somewhat standardised way, that is almost non-invasive. 
Normally CTs have been using infrared light to flash animals during the night, as this was believed to be invisible to the animals (which has later been proven wrong). %TODO
However, the lack of sharpness and detail in these photos limit the information we can retrieve from them (e.g. individual variation in coloration), which has brought us to the usage of white LED flashes.
Naturally, the white LED flash is highly visible for any surface dwelling mammal, which begs the question to what extent it impacts the animals. Or rather, to which \textit{additional} extent it impacts the animals, and therefore, how it affects our data. Animal sightings by CTs can be used to measure species density, and any deviation from the norm in probability of sighting, will skew the precision of the estimate.
\cite{Beddari2019} showed that wolfs (\textit{Canis lupus}) tend to shy away from CTs using white LED flash, whilst the lynx (\textit{Lynx lynx}) is less bothered, compared to the usage of infrared flashes. %TODO vask!
The wolfs were more shy and aware of all CTs in general, attributed to their higher sense of smell, which is a reminder that each species will perceive CT presence different, and thus behave differently as a respone to the stimuli.

\subparagraph{Hypotheses}
In this study, I will attempt to quantify how the usage of white LED flash affects the detection rate of \textsl{the most common large mammal species in the area} and whether this effect correlates with other factors as urbanisation.

* Hypothesis 0: Usage of white LED flash will have no effect on the detection rate of any species.

* Hypothesis 1: Usage of white LED flash will stress one or more species in general, and therefore lower the detection rate of the stressed species. The effect will likely vary in extent between species.

* Hypothesis 2: The effect of the white LED will correlate with urbanisation-factors, as individuals that live closer to urban areas are habituated to Artificial Light At Night (ALAN), and thus will have a weaker response to the white LED


