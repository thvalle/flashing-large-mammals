In an equivalence test, model parameters are tested against a Region of Practical Equivalence (ROPE) as opposed to merely one single mean value, thus accounting for the \emph{effect size} of each parameter.
If the parameters estimate and confidence interval (CI) falls outsied the ROPE, their null hypothesis is rejected. However, if the CI is inside the ROPE, H0 is accepted, no matter if a standard Null Hypothesis Significance Test (NHST) would have deemed it significant.



%Figure TK, which is taken from Lakens 2017, %\cite{Lakens2017}
%demonstrates four different scenarios where model parameters are considered statistically equivalent or not, and statistically different from zero or not. 

