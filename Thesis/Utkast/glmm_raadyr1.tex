Results

GLMM


For roe deer, the model's total explanatory power is substantial (conditional R2 = 0.45) and the part related to the fixed effects alone (marginal R2) is just 0.01.

In other words, most of the variation is due to seasonal changes and variation between the different camera sites. Usage of white LED over time had no significant effect.

The model's intercept, corresponding to time.deploy = 0 and flash = 0, is at -3.38 (p < .001).
That is to say, there were low chances of detecting any roe deer at an IR-camera the rest of the day after I had visited it.


All of the groups are showing a negative, non-significant trend over time.
A negative slope for the control group is strange, as it should represent a baseline detection probability. Any fluctuations in detection rates due to weekly (and ultimately seasonal) changes should be controlled for by the random effect-argument for week of the year. Therefore, I expected the control slope to be close to zero. 

Nevertheless, seeing as all the parameters related to time since deployment are well within the ROPE area in figure \vref{fig:para_raa3} (/ all have a non-significant p-value), it is safe to say that these negative slopes are only due to chance.
When a parameter is within the ROPE area in an equivalence test, it signifies that the difference from the mean, and the variance of the parameter, is low enough that we can accept H0. According to this test, white LED is different enough that we cannot conclude on it's immediate effect (intercept value), but it's trend over time is practically equivalent to H0.

Summing up, I fail to reject H0 for roe deer.

