Results
GLMM

For roe deer, the model explaining variation in detection rate has a total explanatory power that is substantial (conditional R2 = 0.45), but the part related to the fixed effects alone (marginal R2) is just 0.01.

In other words, most of the variation in detection rate is due to seasonal changes and variation between the different camera sites captured in the random terms.
Usage of white LED over time had no significant effect on detection rate of roe deer.

All three treatment groups showed a negative, non-significant trend over time.
As the control-group stayed unchanged through the whole study period, and was visited less than the other cameras, I expected there to be no trend over time (i.e. time.deploy:flashControl \approx 0).

Any fluctuations in detection rates due to weekly (and ultimately seasonal) changes should be controlled for by the random effect-argument for week of the year.
%A negative slope for the control group is strange, as it should represent a baseline detection probability.

Nevertheless, seeing as all the parameters related to time since deployment are well within the Region of Practical Equivalence (ROPE)  in figure X (/ all have a non-significant p-value), indicate that these negative slopes are only due to chance. When a parameter is within the ROPE in an equivalence test, it signifies that the difference from the mean, and the variance of the parameter, is low enough that we can accept H0.
According to this test, white LED is different enough that we cannot conclude on it’s immediate effect (intercept value), but it’s trend over time (interaction with time since deployment) is practically equivalent to H0.
In other words, I haven't found any effect from the usage of white LED on the detection rate of roe deer, and fail to reject H0.
